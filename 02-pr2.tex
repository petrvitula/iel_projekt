\section{Příklad 2}
% Jako parametr zadejte skupinu (A-H)
\druhyZadani{B}

\subsection{Výpočet $R_{i}$}
\begin{figure}[H]
    V obvodu vyzkratujeme zdroj a odstraníme rezistor $R_6$. Následně spočítame napětí mezi svorkami, kde byl původně rezistor $R_6$. 
 

    \begin{circuitikz}
        \draw
        (0,0) -- (0,4)
        (0,4) to[R, l=$R_1$] (3,4)
        (6,4) to[R, l=$R_4$] (6,2)
        (3,2) to[R, l=$R_3$] (3,0)
        (3,4) to[R, l=$R_2$] (3,2)
        (6,2) to[R, l=$R_5$] (6,0)
        (0,0) -- (3,0)
        (3,4) -- (6,4)
        (3,0) -- (6,0)
        (3,4) node[ocirc]{} (3,4)
        (6,0) node[ocirc]{} (6,0)
        (6,2) node[ocirc]{} (6,2)
        (3,0) node[ocirc]{} (3,0)
        {[anchor=west] (6,0) node{B} (6,2) node{A}}
        ;
    \end{circuitikz}

    \begin{equation*}
    \begin{aligned}
      R_{23}  & = R_2 + R_3                                   \\
      R_{123} & = \frac{R_1 \cdot R_{23}}{R_1 + R_{23}}       \\
      R_{1234}& = R_{123} + R_4                               \\
      R_{12345}& = \frac{R_{1234} \cdot R_5}{R_{1234} + R_5}  \\
      R_i & = R_{12345}
    \end{aligned}
  \end{equation*}
\end{figure}

\newpage


\subsection{Výpočet $U_i$}
\begin{figure}[H]
    Zahájíme výpočet $R_{ekv}$ pomocí kterého budeme schopni se dále dopočítat k $I_x$ a $U_i$.

    \begin{circuitikz}
        \draw
        (0,0) to[dcvsource, v<=$U$, , i=$I_x$](0,4)
        (0,4) to[R, l=$R_1$] (3,4)
        (6,4) to[R, l=$R_4$] (6,2)
        (3,2) to[R, l=$R_3$] (3,0)
        (3,4) to[R, l=$R_2$] (3,2)
        (6,2) to[R, l=$R_5$, v=$U_i$] (6,0)
        (0,0) -- (3,0)
        (3,4) -- (6,4)
        (3,0) -- (6,0)
        (3,4) node[ocirc]{} (3,4)
        (6,0) node[ocirc]{} (6,0)
        (6,2) node[ocirc]{} (6,2)
        (3,0) node[ocirc]{} (3,0)
        {[anchor=west] (6,0) node{B} (6,2) node{A}}
        ;
    \end{circuitikz}

    \begin{equation*}
        \begin{aligned}
             R_{45} & = R_4 + R_5                                                 \\
             R_{2345} & = \frac{R_{23} \cdot R_{45}}{R_{23} + R_{45}}             \\
             R_{ekv} & = R_1 + R_{2345}                                           \\
             I_x     & = \frac{U}{R_{ekv}}                                        \\
             U_{2345} &  = U - U_{R1} = U - (R_1 \cdot I_x)                       \\
             I_{45} & = \frac{U_{2345}}{R_{45}}                                   \\
             U_i     & = I_{45} \cdot R_5
        \end{aligned}
    \end{equation*}

\end{figure}

\subsection{Výpočet proudu a napětí na $R_6$}
\begin{figure}[H]
    Máme hodnoty $U_i$ a $R_i$, takže můžeme podle Ohmova zákona snadno spočítat $I_{R1}$ a $U_{R1}$.
    
  \begin{circuitikz}
        \draw
        (0,0) to[dcvsource, v<=$U_i$](0,3)
        (0,3) to[R, l=$R_i$] (3,3)
        (3,3) to[R, l=$R_6$, i=$I_{R6}$, v=$U_{R6}$] (3,0)
        (0,0) -- (3,0)
        ;
  \end{circuitikz}

  
  \begin{equation*}
        \begin{aligned}
             I_{R6} & = \frac{U_i}{R_i + R_6} \\
             U_{R6} & = R_6 \cdot I_{R6}
        \end{aligned}
  \end{equation*}
\end{figure}

\newpage

\subsection{Dosazení}
\begin{figure}[H]
  \begin{equation*}
    \begin{aligned}
         R_{23}  & = R_2 + R_3 = 310 + 610 = 920 \SI{}{\ohm}                                \\
         R_{123}  & = \frac{R_1 \cdot R_{23}}{R_1 + R_{23}} = \frac{50 \cdot 920}{50 + 920} = \frac{46000}{970} = 47.4227 \SI{}{\ohm}                                                            \\
         R_{1234}  & = R_{123} + R_4 = 47.4227 + 220 = 267.4227 \SI{}{\ohm}                     \\
         R_{12345}  & = \frac{R_{1234} \cdot R_5}{R_{1234} + R_5} = \frac{267.4227 \cdot 570}{267.4227 + 570} = \frac{152430,939}{837.4227} = 182.0239 \SI{}{\ohm}                                 \\
         R_i &  = 182.0239 \SI{}{\ohm}                                                                 \\
         R_{45} & = R_4 + R_5 = 220 + 570 = 790\SI{}{\ohm}                                               \\
         R_{2345} & = \frac{R_{23} \cdot R_{45}}{R_{23} + R_{45}} = \frac{920 \cdot 790}{920 + 790} = \frac{726800}{1710} = 425.0292 \SI{}{\ohm}                                                  \\
         R_{ekv} & = R_1 + R_{2345} = 50 + 25.0292 = 475.0292 \SI{}{\ohm}                           \\
         I_x & = \frac{U}{R_{ekv}} = \frac{100}{475.0292} = 0.2105 A                                \\
         U_{2345} & = U - (R_1 \cdot I_x) = 100 - (50 \cdot 0.2105) = 89.4743 V                        \\
         I_{45} & = \frac{U_{2345}}{R_{45}} = \frac{89.4743}{790} = 0.1133 A                            \\
         U_i & = I_{45} \cdot R_5 = 0.1133 \cdot 570 = 64.5574 V                                         \\
         I_{R6} & = \frac{U_i}{R_i + R_6} = \frac{64.5574}{182.0239 + 100} = \underline{\underline{0.2289 A}}      \\
         U_{R6} & = R_6 \cdot I_{R6} = 100 \cdot 0.2289 = \underline{\underline{22.8908 V}}             \\
    \end{aligned}
  \end{equation*}
\end{figure}